\documentclass{ltxdoc}
\input{preamble.tex}
\usepackage{shortvrb}
\MakeShortVerb{\|} % Short verb, use |something| instead of \verb|something|

\title{TheFire888's \LaTeX\ Cheatsheet}
\author{\textit{luanleal@usp.br}}
\date{}

\begin{document}
\maketitle

Esse \textit{cheatsheet} assume que você está usando meu
\texttt{preamble.tex}. Se você não sabe onde achar esse documento, me
mande um email.

\section{O pacote \texttt{siunitx}}
O pacote \texttt{siunitx} é uma ferramenta indispensável no LaTeX para
formatação precisa de unidades, valores numéricos e grandezas
científicas, seguindo as convenções do Sistema Internacional (SI). O
ponto chave desse pacote é a forma consistente e profissional de
formatar unidades e números, o próprio autor do pacote afirma:
``\textit{In this way, users can use siunitx to follow the
    requirements of publishers, co-authors, universities etc. without
    needing to alter the input at all.}''. Vejamos seus usos:

\subsection{Unidades}
A macro principal para unidades é |\qty{}{}|, mas o pacote também tem
macros como |\qtylist{}{}|, |\qtyrange{}{}{}| e
|\unit{}|. \\

\DescribeMacro{\qty{}{}} Uso: |\qty{number}{unit}|

Essa será a macro que você mais deverá utilizar dessa
biblioteca, ela permite combinar um valor numérico e uma unidade. Você
pode usar |\pm| para indicar incertezas; |\squared| para elevar a
unidade que precede o comando ao quadrado; |\cubic| para elevar a
unidade que sucede o comando ao cubo. Note que a diferença entre
|\squared| e |\cubic| se deve a forma como essas expressões são
escritas em inglês. Para te poupar tempo explicando outras macros,
veja alguns exemplos na \cref{xpqty}.

\begin{table}[!htb]
\caption{Exemplos de uso do \texttt{\textbackslash qty}}\label{xpqty}
\centering
\begin{tabular}{l c}
\toprule
Comando & Resultado \\
\midrule
|\qty{2.5}{\milli\mol}| & \qty{2.5}{\milli\mol} \\
|\qty{9.81}{\meter\per\second\squared}| & \qty{9.81}{\meter\per\second\squared} \\
|\qty{1.21\pm0.12}{\kilo\gram\cubic\meter\per\second}| &
\qty{1.21\pm0.12}{\kilo\gram\cubic\meter\per\second} \\
|\qty{300}{\kilo\meter\per\hour}| & \qty{300}{\kilo\meter\per\hour} \\
|\qty{1.4}{\mega\pascal\squared}| & \qty{1.4}{\mega\pascal\squared} \\
|\qty{5.0e-3}{\newton\per\meter}| & \qty{5.0e-3}{\newton\per\meter} \\
|\qty{22.4\pm0.02}{\liter\per\mol}| & \qty{22.4\pm0.02}{\liter\per\mol} \\
|\qty{21}{\kg}| & \qty{21}{\kg}\\
|\qty{0.2}{\ul}| & \qty{0.2}{\ul}\\
\bottomrule
\end{tabular}
\end{table}

Agora, o leitor deve se perguntar sobre todas as outras macros
disponibilizadas por esse pacote para especificar unidades. Nesse
caso, recomendo a leitura da documentação que acompanha esse pacote no
CTAN. Inclusive, recomendo ao leitor que SEMPRE procure no CTAN quando
surgir uma dúvida sobre o uso de um pacote no \LaTeX, isso irá te
economizar muito tempo em fóruns.\\

\DescribeMacro{\qtylist} Uso: |\qtylist{n1;n2;...}{unit}|

Essa macro serve unicamente para expressar listas de números seguidos
de unidades. Seu uso se assemelha ao uso de |\qty|. Veja alguns
exemplos na \cref{xpqtylist}.\\

\begin{table}[!htb]
\caption{Exemplos de uso do \texttt{\textbackslash qtylist}}\label{xpqtylist}
\centering
\begin{tabular}{l c}
\toprule
Comando & Resultado \\
\midrule
|\qtylist{1;2;3}{\ul}| & \qtylist{1;2;3}{\ul} \\
|\qtylist{2.5\pm0.4;1.2\pm0.2}{\astronomicalunit}| &
\qtylist{2.5\pm0.4;1.2\pm0.2}{\astronomicalunit} \\
|\qtylist{0.5;1.0;1.5}{\weber}| & \qtylist{0.5;1.0;1.5}{\weber} \\
|\qtylist{-10;0;25;100}{\celsius}| & \qtylist{-10;0;25;100}{\celsius} \\
|\qtylist{273.15;300;500}{\kelvin}| & \qtylist{273.15;300;500}{\kelvin} \\
|\qtylist{1000;2000;5000}{\dalton}| & \qtylist{1000;2000;5000}{\dalton} \\
|\qtylist{1.2\pm0.1;3.4\pm0.2}{\tesla}| & \qtylist{1.2\pm0.1;3.4\pm0.2}{\tesla} \\
|\qtylist{5;10;15}{\mmHg}| & \qtylist{5;10;15}{\mmHg} \\
\bottomrule
\end{tabular}
\end{table}

\DescribeMacro{\qtyrange} Uso: |\qtyrange{n1}{n2}{unit}|

Essa macro serve para expressar intervalos de números seguidos de
unidades. Seu uso também se assemelha ao uso de |\qty|. Veja alguns
exemplos na \cref{xpqtyrange}.

\begin{table}[!htb]
\caption{Exemplos de uso do \texttt{\textbackslash qtyrange} com incertezas}\label{xpqtyrange}
\centering
\begin{tabular}{l c}
\toprule
Comando & Resultado \\
\midrule
|\qtyrange{15.2\pm0.3}{25.7\pm0.5}{\percent}| & \qtyrange{15.2\pm0.3}{25.7\pm0.5}{\percent} \\
|\qtyrange{1.25\pm0.01}{2.50\pm0.02}{\milli\mole}| & \qtyrange{1.25\pm0.01}{2.50\pm0.02}{\milli\mole} \\
|\qtyrange{3.0\pm0.2}{5.0\pm0.3}{\kilo\newton}| & \qtyrange{3.0\pm0.2}{5.0\pm0.3}{\kilo\newton} \\
|\qtyrange{100\pm5}{200\pm10}{\lux}| & \qtyrange{100\pm5}{200\pm10}{\lux} \\
|\qtyrange{0.5\pm0.1}{1.5\pm0.2}{\becquerel}| & \qtyrange{0.5\pm0.1}{1.5\pm0.2}{\becquerel} \\
|\qtyrange{2.4\pm0.1}{3.6\pm0.2}{\electronvolt}| & \qtyrange{2.4\pm0.1}{3.6\pm0.2}{\electronvolt} \\
|\qtyrange{0.8\pm0.05}{1.2\pm0.08}{\gray\per\second}| & \qtyrange{0.8\pm0.05}{1.2\pm0.08}{\gray\per\second} \\
|\qtyrange{50\pm2}{100\pm5}{\siemens\per\centi\meter}| & \qtyrange{50\pm2}{100\pm5}{\siemens\per\centi\meter} \\
\bottomrule
\end{tabular}
\end{table}

\DescribeMacro{\unit} Uso: |\unit{unit}|

Em alguns casos especiais, você pode querer expressar unidades
sozinhas. Nesses casos, você pode usar essa macro. Veja alguns
exemplos na \cref{xpunit}.

\begin{table}[!htb]
\caption{Exemplos de uso do \texttt{\textbackslash unit}}\label{xpunit}
\centering
\begin{tabular}{l c}
\toprule
Comando & Resultado \\
\midrule
|\unit{\meter\per\second}| & \unit{\meter\per\second} \\
|\unit{\kilo\gram\cdot\meter\squared}| & \unit{\kilo\gram\cdot\meter\squared} \\
|\unit{\micro\mol\per\liter}| & \unit{\micro\mol\per\liter} \\
|\unit{\ohm\cdot\meter}| & \unit{\ohm\cdot\meter} \\
|\unit{\degreeCelsius}| & \unit{\degreeCelsius} \\
|\unit{\newton\second}| & \unit{\newton\second} \\
|\unit{\farad\per\meter}| & \unit{\farad\per\meter} \\
|\unit{\becquerel}| & \unit{\becquerel} \\
|\unit{\henry\per\second}| & \unit{\henry\per\second} \\
|\unit{\pascal\second}| & \unit{\pascal\second} \\
|\unit{\joule\per\kelvin}| & \unit{\joule\per\kelvin} \\
|\unit{\watt\per\square\meter}| & \unit{\watt\per\square\meter} \\
\bottomrule
\end{tabular}
\end{table}

\section{O pacote \texttt{amsmath}}
O pacote |amsmath| é a referência padrão para formatação de conteúdo
matemático em LaTeX, desenvolvido pela American Mathematical Society
(AMS). Ele fornece estruturas avançadas para escrita matemática
profissional, sendo amplamente adotado em publicações acadêmicas,
livros técnicos e artigos científicos. Em geral, vamos utilizar esse
pacote junto ao pacote |amssymb|, de forma que esse guia não fará
distinção desses pacotes. Se for do seu interesse fazer tal distinção,
consulte o CTAN.

Entre suas principais funcionalidades estão os ambientes |align|, |gather| e |multline|, que permitem criar equações alinhadas, agrupadas ou com quebras de linha, respectivamente. O ambiente |align| é particularmente útil para alinhar múltiplas equações em pontos específicos, como sinais de igual, usando o caractere |&| como marcador de alinhamento. Além disso, |amsmath| introduz comandos como |\intertext{}| para inserir texto entre equações sem perturbar o alinhamento, e |\tag{}| para personalizar a numeração de equações. O pacote também inclui ambientes especializados para matrizes, como |matrix|, |pmatrix| e |bmatrix|, que simplificam a criação de matrizes com diferentes delimitadores.
\\


\DescribeEnv{align} Esse ambiente permite escrever equações com
alinhamento vertical preciso. Ele permite alinhar múltiplas equações
em relação a um ponto específico (geralmente o sinal de igual) usando
o caractere |&| como marcador de alinhamento. Cada linha é
automaticamente numerada, exceto quando se usa |align*|, no lugar de
|align|.

Vejamos um exemplo de equação:\\
\begin{minipage}{0.3\textwidth}
\begin{verbatim}
\begin{align}
    f(x) &= x^2 + 2x \\
         &= x (x + 2)
\end{align}
\end{verbatim}
\end{minipage}
\begin{minipage}{0.3\textwidth}
\begin{align}
    f(x) &= x^2 + 2x \\
         &= x (x + 2)
\end{align}
\end{minipage}\\

Um resultado parecido poderia ser obtido por uso do |align*|, nesse
caso as linhas não serão numeradas:\\

\begin{minipage}{0.3\textwidth}
\begin{verbatim}
\begin{align*}
    f(x) &= x^2 + 2x \\
         &= x (x + 2)
\end{align*}
\end{verbatim}
\end{minipage}
\begin{minipage}{0.3\textwidth}
\begin{align*}
    f(x) &= x^2 + 2x \\
         &= x (x + 2)
\end{align*}
\end{minipage}\\

\end{document}
